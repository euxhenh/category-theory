\section{Adjoints, representables and limits}

\subsection{Limits in terms of representables and adjoints}
Consider categories $\bcat{I}$ and $\cat{A}$ and fix an object $A\in\cat{A}$. There exists a functor $\Delta A:\bcat{I}\to\cat{A}$ which sends every object of $\bcat{I}$ to $A$ and every map to $1_A$. This defines a \textbf{diagonal functor}
\begin{equation*}
    \Delta : \cat{A}\to [\bcat{I},\cat{A}].
\end{equation*}
When $\bcat{I}$ is discrete and has two objects, then $[\bcat{I},\cat{A}]=\cat{A}\times\cat{A}$ and $\Delta(A)=(A,A)$.

Now, consider a diagram $D:\bcat{I}\to\cat{A}$. A cone on $D$ with vertex $A$ has maps going from $A$ into $D(\bcat{I})$ which satisfy composition criteria. This can be thought of as a natural transformation from $\Delta A$ to $D$
\begin{equation*}
\begin{tikzcd}
    \bcat{I} \arrow[r, bend left=30, sloped, "\Delta A"{name=t}]
    \arrow[r, bend right=30, sloped, "D"'{name=b}]
    & \cat{A} \
    \arrow[from=t, Rightarrow, to=b, "", shorten <= 3pt, shorten >= 3pt]
\end{tikzcd}
\end{equation*}

If we denote the set of cones on $D$ with vertex $A$ by $\text{Cone}(A,D)$, we have
\begin{equation*}
    \text{Cone}(A,D)=[\bcat{I},\cat{A}](\Delta A, D).
\end{equation*}
This means that for every natural transformation we get a different cone. Thus, $\text{Cone}(A,D)$ is functorial in $A$ (contravariantly) and $D$ (covariantly).

\begin{proposition}
    Let $\bcat{I}$ be a small category, $\cat{A}$ a category, and $D:\bcat{I}\to\cat{A}$ a diagram. Then there is a one-to-one correspondence between limit cones on $D$ and representations of the functor
    \begin{equation*}
        \textup{Cone}(-,D): \catop{A}\to\bcat{Set},
    \end{equation*}
    with the representing objects of $\textup{Cone}(-,D)$ being the limit objects (that is, the vertices of the limit cones) of $D$.
\end{proposition}

In other words, given an object $A\in\cat{A}$, if $D$ has a limit then
\begin{equation*}
    \text{Cone}(A,D)\cong \cat{A}\Big( A, \lim D \Big)
\end{equation*}
naturally in $A$.

\begin{corollary}
    Limits are unique up to isomorphism.
\end{corollary}

\begin{lemma}
    Let $\bcat{I}$ be a small category and \begin{tikzcd}
        \bcat{I} \arrow[r, bend left=30, sloped, "D"{name=t}]
        \arrow[r, bend right=30, sloped, "D'"'{name=b}]
        & \cat{A} \
        \arrow[from=t, Rightarrow, to=b, "\alpha", shorten <= 3pt, shorten >= 3pt]
    \end{tikzcd} a natural transformation. Let
    \begin{equation*}
        \Big( \lim D\xrightarrow{p_I} D(I) \Big)_{I\in\bcat{I}} \qquad \text{and} \qquad \Big( \lim D'\xrightarrow{p_I'} D'(I) \Big)_{I\in\bcat{I}}
    \end{equation*}
    be limit cones. Then:
    \begin{enumerate}[label=(\alph*)]
        \item there is a unique map $\lim\alpha: \lim D\to\lim D'$ such that for all $I\in\bcat{I}$, the square
            \begin{equation*}
            \begin{tikzcd}
                \lim D \arrow[r, "p_I"] \arrow[d, "\lim\alpha"'] & D(I) \arrow[d, "\alpha_I"] \\
                \lim D' \arrow[r, "p_I'"'] & D'(I)
            \end{tikzcd}
            \end{equation*}
            commutes;
        \item given cones $\Big( A\xrightarrow{f_I} D(I) \Big)_{I\in\bcat{I}}$ and $\Big( A'\xrightarrow{f_I'} D'(I)\Big)_{I\in\bcat{I}}$ and a map $s:A\to A'$ such that
            \begin{equation*}
            \begin{tikzcd}
                A \arrow[r, "f_I"] \arrow[d, "s"'] & D(I) \arrow[d, "\alpha_I"] \\
                A' \arrow[r, "f'_I"'] & D'(I)
            \end{tikzcd}
            \end{equation*}
            commutes for all $I\in\bcat{I}$, the square
            \begin{equation*}
            \begin{tikzcd}
                A \arrow[r, "\overline{f}"] \arrow[d, "s"'] & \lim{D} \arrow[d, "\lim{\alpha}"] \\
                A' \arrow[r, "\overline{f'}"'] & \lim{D'}
            \end{tikzcd}
            \end{equation*}
            also commutes.
    \end{enumerate}
\end{lemma}

\begin{proposition}
    Let $\bcat{I}$ be a small category and $\cat{A}$ a category with all limits of shape $\bcat{I}$. Then $\lim$ defines a functor $[\bcat{I},\cat{A}]\to\cat{A}$, and this functor is right adjoint to the diagonal functor.
\end{proposition}

\subsection{Limits and colimits of presheaves}
\begin{center}
    \textbf{Representables preserve limits}
\end{center}
By definition of product, a map $A\to X\times Y$ amounts to a pair of maps $(A\to X, A\to Y)$, where $A,X$ and $Y$ are objects of a category $\cat{A}$ with binary products. There is therefore a bijection
\begin{equation*}
    \cat{A}(A,X\times Y)\cong \cat{A}(A,X)\times\cat{A}(A,Y)
\end{equation*}
naturally in $A,X,Y\in\cat{A}$.\par

Similarly, if we consider the fork \begin{equation*}
\begin{tikzcd}
    E \arrow[r, "e"] & X \arrow[r, shift left, "s"] \arrow[r, shift right, "t"'] & Y
\end{tikzcd}
\end{equation*}
where $E$ is the equalizer of $s$ and $t$, any map from $A$ to $E$ corresponds one-to-one with maps $f:A\to X$ such that $s\circ f=t\circ f$. Now $s$ induces a map
\begin{equation*}
    s_*=\cat{A}(A,s): \cat{A}(A,X)\to\cat{A}(A,Y),
\end{equation*}
and similarly for $t$. In this notation, what we have just said is that maps from $A$ to $E$ correspond one-to-one with elements $f\in\cat{A}(A,X)$ such that
\begin{equation*}
    \big( \cat{A}(A,s) \big)(f)=\big( \cat{A}(A,t) \big)(f).
\end{equation*}
But such an $f$ is exactly an element of the equalizer of $\cat{A}(A,s)$ and $\cat{A}(A,t)$, so we have a canonical bijection
\begin{equation*}
    \cat{A}\Big( A, \text{Eq}\big( \begin{tikzcd}
        X \arrow[r, shift left, "s"] \arrow[r, shift right, "t"'] & Y
\end{tikzcd} \big) \Big) \cong \text{Eq}\Big( \begin{tikzcd}
        \cat{A}(A,X) \arrow[r, shift left, "\cat{A}(A{,}s)"] \arrow[r, shift right, "\cat{A}(A{,}t)"'] & \cat{A}(A,Y)
\end{tikzcd} \Big).
\end{equation*}

This looks similar to the isomorphism for products. They suggest that, more generally, we might have the following:

\begin{lemma}
    Let $\bcat{I}$ be a small category, $\cat{A}$ a locally small category, $D:\bcat{I}\to\cat{A}$ a diagram, and $A\in\cat{A}$. Then
    \begin{equation*}
        \textup{Cone}(A,D)\cong\lim{\cat{A}(A,D)}
    \end{equation*}
    naturally in $A$ and $D$.
\end{lemma}

\begin{proposition}[Representables preserve limits]
    Let $\cat{A}$ be a locally small category and $A\in\cat{A}$. Then $\cat{A}(A,-): \cat{A}\to\bcat{Set}$ preserves limits.
\end{proposition}
\begin{proof}
    Let $\bcat{I}$ be a small category and let $D:\bcat{I}\to\cat{A}$ be a diagram that has a limit. Then
    \begin{equation*}
        \cat{A}{\Big( A,\lim{D} \Big)} \cong \text{Cone}(A,D) \cong \lim{\cat{A}(A,D)}
    \end{equation*}
    naturally in $A$.
\end{proof}

To dualize this proposition, we replace $\cat{A}$ by $\catop{A}$. Thus, $\cat{A}(-,A):\catop{A}\to\bcat{Set}$ preserves limits. A limit in $\catop{A}$ is a colimit in $\cat{A}$, so $\cat{A}(-,A)$ transforms colimits in $\cat{A}$ into limits in $\bcat{Set}$.\par

For example, let $X,Y$ and $A$ be objects of a category $\cat{A}$, and suppose the sum $X+Y$ exists. There is a canonical isomorphism
\begin{equation*}
    \cat{A}(X+Y,A) \cong \cat{A}(X,A)\times \cat{A}(Y,A).
\end{equation*}

\begin{center}
    \textbf{Limits in functor categories}
\end{center}
We shall analyze limits and colimits in functor categories $[\bcat{A},\cat{S}]$, where $\bcat{A}$ is small and $\cat{S}$ is locally small. The important cases are $\cat{S}=\bcat{Set}$ and $\cat{S}=\bcatop{Set}$. Limits and colimits in $[\bcat{A},\cat{S}]$ work in a simple way. For instance, if $\cat{S}$ has binary products then so does $[\bcat{A}, \cat{S}]$, and the product of two functors $X,Y:\bcat{A}\to\cat{S}$ is the functor $X\times Y:\bcat{A}\to\cat{S}$ given by
\begin{equation*}
    (X\times Y)(A) = X(A)\times Y(A)
\end{equation*}
for all $A\in\bcat{A}$.

\begin{notation}
    Let $\bcat{A}$ and $\cat{S}$ be categories. For each $A\in\bcat{A}$, there is a functor
    \begin{align*}
        \text{ev}_A: [\bcat{A},\cat{S}] &\to \cat{S}\\
        X &\mapsto X(A),
    \end{align*}
    called \textbf{evaluation} at $A$. We will be working with diagrams in $[\bcat{A},\cat{S}]$, and given such a diagram $D:\bcat{I}\to[\bcat{A},\cat{S}]$, we have for each $A\in\bcat{A}$ a functor
    \begin{align*}
        \text{ev}_A\circ D: \bcat{I} &\to \cat{S}\\
        I &\mapsto D(I)(A).
    \end{align*}
    We write $\text{ev}_A\circ D$ as $D(-)(A)$.
\end{notation}

\begin{theorem}[Limits in functor categories]
    Let $\bcat{A}$ and $\bcat{I}$ be small categories and $\cat{S}$ a locally small category. Let $D:\bcat{I}\to[\bcat{A},\cat{S}]$ be a diagram, and suppose that for each $A\in\bcat{A}$, the diagram $D(-)(A):\bcat{I}\to\cat{S}$ has a limit. Then there is a cone on $D$ whose image under $\text{ev}_A$ is a limit cone on $D(-)(A)$ for each $A\in\bcat{A}$. Moreover, any such cone on $D$ is a limit cone.
\end{theorem}
This can be expressed as
\begin{center}
    \textit{Limits in a functor category are computed pointwise}.
\end{center}

\begin{corollary}
    Let $\bcat{I}$ and $\bcat{A}$ be small categories, and $\cat{S}$ a locally small category. If $\cat{S}$ has all limits (respectively, colimits) of shape $\bcat{I}$ then so does $[\bcat{A},\cat{S}]$, and for each $A\in\bcat{A}$, the evaluation functor $\text{ev}_A:[\bcat{A},\cat{S}]\to\cat{S}$ preserves them.
\end{corollary}

\begin{corollary}
    Let $\bcat{A}$ be a small category. Then $[\bcatop{A}, \bcat{Set}]$ has all limits and colimits, and for each $A\in\bcat{A}$, the evaluation functor $\text{ev}_A:[\bcatop{A},\bcat{Set}]\to\bcat{Set}$ preserves them.
\end{corollary}

\begin{corollary}\label{cor:yonedalimit}
    The Yoneda embedding $H_{\bullet}:\bcat{A}\to[\bcatop{A},\bcat{Set}]$, preserves limits, for any small category $\bcat{A}$.
\end{corollary}

\begin{example}
    Let $\bcat{A}$ be a category with binary products. Corollary \ref{cor:yonedalimit} implies that for all $X,Y\in\bcat{A}$,
    \begin{equation*}
        H_{X\times Y}\cong H_X\times H_Y
    \end{equation*}
    in $[\bcatop{A},\bcat{Set}]$.
\end{example}
