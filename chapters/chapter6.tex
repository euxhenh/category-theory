\section{Limits}
\subsection{Definition and examples}
\begin{center}
    \textbf{Products}
\end{center}
Let $X$ and $Y$ be sets. The familiar cartesian product $X\times Y$ consists of pairs of elements from $X$ and $Y$.

\begin{definition}
    Let $\cat{A}$ be a category and $X,Y\in\cat{A}$. A \textbf{product} of $X$ and $Y$ consists of an object $P$ and maps
    \begin{equation*}
    \begin{tikzcd}
        & P \arrow[dl, "p_1"'] \arrow[dr, "p_2"] \\
        X && Y
    \end{tikzcd}
    \end{equation*}
    with the property that for all objects and maps
    \begin{equation*}
    \begin{tikzcd}
        & A \arrow[dl, "f_1"'] \arrow[dr, "f_2"] \\
        X && Y
    \end{tikzcd}
    \end{equation*}
    in $\cat{A}$, there exists a unique map $\bar{f}:A\to P$ such that
    \begin{equation*}
    \begin{tikzcd}
        & A \arrow[ddl, "f_1"'] \arrow[ddr, "f_2"] \arrow[d, dotted, "\bar{f}" description] \\
        & P \arrow[dl, "p_1"] \arrow[dr, "p_2"'] \\
        X && Y
    \end{tikzcd}
    \end{equation*}
    commutes. The maps $p_1$ and $p_2$ are called the \textbf{projections}.
\end{definition}

\begin{example}
    Any two sets $X$ and $Y$ have a product in \textbf{Set}, however, products do not always exist. It is easy to check that the usual cartesian product $X\times Y$, is really a product in the sense of the definition above.
\end{example}

\begin{example}
    In the category of topological spaces, any two objects $X$ and $Y$ have a product. It is the set $X\times Y$ equipped with the product topology and the standard projection maps. The product topology is designed so that the function
    \begin{align*}
        A &\to X\times Y\\
        t &\mapsto \big(x(t), y(t)\big)
    \end{align*}
    is continuous if and only if both functions
    \begin{equation*}
        t\mapsto x(t), \qquad t\mapsto y(t)
    \end{equation*}
    are continuous.\par

    The product topology is the crudest topology on $X\times Y$ for which the projections are continuous. In this sense, if we have another topology $\mathcal{T}$ on $X\times Y$ such that $p_1$ and $p_2$ are continuous, then every subset of $X\times Y$ which is open in the product topology, is also open in $\mathcal{T}$. I.e., there exists a unique map from $\mathcal{T}$ to the product topology which is continuous.
\end{example}

\begin{example}
    View the poset $(\mathbb{R}, \leq)$ as a category. Then the product of $x,y\in\mathbb{R}$ is $\min\{x,y\}$. Indeed, we have $\min\{x,y\}\leq x$ and $\min\{x,y\}\leq y$ and for all $a\in\mathbb{R}$ which satisfy $a\leq x$, and $a\leq y$, we must have $a\leq\min\{x,y\}$.
\end{example}

In a similar fashion, if $S$ is a set, one can view $X\cap Y$ as the product of $X,Y\in\powerset{S}$ in the poset $(\powerset{S}, \subseteq)$ regarded as a category. And $\gcd(x,y)$ as the product of $x,y\in\mathbb{N}$ in the poset $(\mathbb{N}, \vert)$ regarded as category.\par

In general, when a poset is regarded as a category, meets are exactly products. They do not always exist, but when they do, they are unique.\par

The product of a family of objects can be constructed in the most obvious way.\par

Let $\cat{A}$ be a category, and consider an $I$-indexed family $(X_i)_{i\in I}$ of objects of $\cat{A}$ which is a function $I\to\text{ob}(\cat{A})$. The product of the empty family consists of an object $P$ of $\cat{A}$ such that for each object $A\in\cat{A}$, there exists a unique map $\bar{f}:A\to P$. (The commutativity conditions hold trivially.) In other words, a product of the empty family is exactly a terminal object.

\begin{center}
    \textbf{Equalizer}
\end{center}
A \textbf{fork} in a category consists of objects and maps
\begin{equation}\label{eq:fork}
\begin{tikzcd}
    A \arrow[r, "f"] & X \arrow[r, shift left, "s"] \arrow[r, shift right, "t"'] & Y
\end{tikzcd}
\end{equation}
such that $sf=tf$.

\begin{definition}
    Let $\cat{A}$ be a category and let \begin{tikzcd}
        X \arrow[r, shift left, "s"] \arrow[r, shift right, "t"'] & Y
    \end{tikzcd} be objects and maps in $\cat{A}$. An \textbf{equalizer} of $s$ and $t$ is an object $E$ together with a map $E\xrightarrow{i}X$ such that
    \begin{equation*}
        \begin{tikzcd}
            E \arrow[r, "i"] & X \arrow[r, shift left, "s"] \arrow[r, shift right, "t"'] & Y
        \end{tikzcd}
    \end{equation*}
    is a fork, and with the property that for any fork as in (\ref{eq:fork}), there exists a unique map $\bar{f}:A\to E$ such that
    \begin{equation*}
    \begin{tikzcd}
        A \arrow[dd, dotted, "\bar{f}"description] \arrow[ddr, "f"] \\ \\
        E \arrow[r, "i"'] & X
    \end{tikzcd}
    \end{equation*}
    commutes.
\end{definition}

\begin{example}
    In $\bcat{Set}$ an equalizer describes the set of solutions of a single equation. Given sets and functions \begin{tikzcd}
        X \arrow[r, shift left, "s"] \arrow[r, shift right, "t"'] & Y
    \end{tikzcd}, write
    \begin{equation*}
        E = \{ x\in X \mid s(x)=t(x) \},
    \end{equation*}
    and write $i:E\to X$ for the inclusion. Then $si=ti$, so we have a fork, and one can check that it is universal among all forks on $s$ and $t$.
\end{example}

\begin{example}
    Let $\theta:G\to H$ be a homomorphism of groups. Then $\theta$ gives rise to a fork
    \begin{equation*}
    \begin{tikzcd}
        \text{ker}\theta \arrow[r, "\iota"] & G \arrow[r, shift left, "\theta"] \arrow[r, shift right, "\epsilon"'] & H
    \end{tikzcd}
    \end{equation*}
    where $\iota$ is the inclusion and $\epsilon$ is the trivial homomorphism. This is an equalizer in $\bcat{Grp}$. Showing this amounts to showing that the map that we have been calling $\bar{f}$ is a homomorphism.\par

Thus, kernels are a special case of equaliers.
\end{example}

\begin{center}
    \textbf{Pullbacks}
\end{center}

\begin{definition}
    Let $\cat{A}$ be a category, and take objects and maps
    \begin{equation*}
    \begin{tikzcd}
        & Y \arrow[d, "t"] \\
        X \arrow[r, "s"] & Z
    \end{tikzcd}
    \end{equation*}
    in $\cat{A}$. A \textbf{pullback} of this diagram is an object $P\in\cat{A}$ together with maps $p_1:P\to X$ and $p_2:P\to Y$ such that
    \begin{equation*}
    \begin{tikzcd}
        P \arrow[r, "p_2"] \arrow[d, "p_1"'] & Y \arrow[d, "t"] \\
        X \arrow[r, "s"'] & Z
    \end{tikzcd}
    \end{equation*}
    commutes, and with the property that for any commutative square
    \begin{equation*}
    \begin{tikzcd}
        A \arrow[r, "f_2"] \arrow[d, "f_1"'] & Y \arrow[d, "t"] \\
        X \arrow[r, "s"'] & Z
    \end{tikzcd}
    \end{equation*}
    in $\cat{A}$, there is a unique map $\bar{f}:A\to P$ such that
    \begin{equation*}
    \begin{tikzcd}
        A \arrow[dr, dotted, "\bar{f}"description] \arrow[drr, bend left, "f_2"] \arrow[ddr, bend right, "f_1"'] \\
        & P \arrow[r, "p_2"] \arrow[d, "p_1"'] & Y \arrow[d, "t"] \\
        & X \arrow[r, "s"'] & Z
    \end{tikzcd}
    \end{equation*}
    commutes. I.e., $p_1\bar{f}=f_1$ and $p_2\bar{f}=f_2$, since the commutativity of the square is given.
\end{definition}

\begin{remark}
    When $Z$ is a terminal object, then the pullback of $X$ and $Y$ is the product $X\times Y$.
\end{remark}

\begin{example}
    The pullback of a diagram in $\bcat{Set}$
    \begin{equation*}
        P = \{ (x,y) \in X\times Y\mid s(x)=t(y) \}
    \end{equation*}
    with projections $p_1$ and $p_2$ given by $p_1(x,y)=x$ and $p_2(x,y)=y$.
\end{example}

\begin{example}
    Intersection of subsets provides another exmaple of pullbacks. Indeed, let $X$ and $Y$ be subsets of a set $Z$. Then,
    \begin{equation*}
        \begin{tikzcd}
            X\cap Y \arrow[r, hook, ""] \arrow[d, hook, ""'] & Y \arrow[d, hook, ""] \\
            X \arrow[r, hook, ""'] & Z
        \end{tikzcd}
    \end{equation*}
    is a pullback square, where all the arrows are inclusions of subsets.
\end{example}

\begin{center}
    \textbf{The definition of a limit}
\end{center}
All three constructions: products, equalizers, and pullbacks have something in common. In each, we aim to construct an object with a universal property. In each such construction we start with some data.\par

For products, it is a pair of objects
\begin{equation*}
    X \qquad Y.
\end{equation*}
Call this diagram $\bcat{T}$. For equalizers, it is a diagram
\begin{equation*}
    \begin{tikzcd}
        X \arrow[r, shift left, "s"] \arrow[r, shift right, "t"'] & Y
    \end{tikzcd}.
\end{equation*}
Call it $\bcat{E}$. For pullbacks, it is a diagram
\begin{equation*}
\begin{tikzcd}
    & Y \arrow[d, "t"] \\
    X \arrow[r, "s"'] & Z.
\end{tikzcd}
\end{equation*}
Call it $\bcat{P}$.\par

If we view the above diagrams as categories $\bcat{I}$, then a functor $\bcat{I}\to\cat{A}$ corresponds to data of the same shape in $\cat{A}$.

\begin{definition}
    Let $\cat{A}$ be a category and $\bcat{I}$ a small category. A functor $\bcat{I}\to\cat{A}$ is called a \textbf{diagram} in $\cat{A}$ of \textbf{shape} $\bcat{I}$.
\end{definition}

\begin{definition}
    Let $\cat{A}$ be a category, $\bcat{I}$ a small category, and $D:\bcat{I}\to\cat{A}$ a diagram of $\cat{A}$.
    \begin{enumerate}[label=(\alph*)]
        \item A \textbf{cone} on $D$ is an object $A\in\cat{A}$ (the \textbf{vertex} of the cone) together with a family
            \begin{equation*}
                \Big( A\xrightarrow{f_I} D(I) \Big)_{I\in\bcat{I}}
            \end{equation*}
            of maps in $\cat{A}$ such that for all maps $I\xrightarrow{u}J$ in $\bcat{I}$, the triangle
            \begin{equation*}
            \begin{tikzcd}
                & D(I) \arrow[dd, "Du"] \\
                A \arrow[ur, "f_I"] \arrow[dr, "f_J"'] \\
                & D(J)
            \end{tikzcd}
            \end{equation*}
            commutes.
    \item A \textbf{limit} of $D$ is a cone $\Big( L\xrightarrow{p_I} D(I) \Big)_{I\in\bcat{I}}$ with the property that for any cone on $D$, there exists a unique map $\bar{f}:A\to L$ such that $p_I\circ\bar{f}=f_I$ for all $I\in\bcat{I}$. The maps $p_I$ are called the \textbf{projections} of the limit.
    \end{enumerate}
\end{definition}

\begin{remark}
    We sometimes abuse language by referring to $L$ as the limit of $D$. We write $L=\lim{D}$.
\end{remark}

\begin{example}
    Let $\cat{A}$ be any category. Recall the diagrams $\bcat{T}, \bcat{E}$ and $\bcat{P}$ viewed as categories.
    \begin{enumerate}[label=(\alph*)]
        \item A diagram $D$ of shape $\bcat{T}$ in $\cat{A}$ is a pair $(X,Y)$ of objects in $\cat{A}$. A cone on $D$ is an object $A$ with maps $f_1:A\to X$ and $f_2:A\to Y$, and a limit of $D$ is a product of $X$ and $Y$.\par
            In particular, a limit of the unique functor $\emptyset\to\cat{A}$ is a terminal object of $\cat{A}$.
        \item A diagram $D$ of shape $\bcat{E}$ in $\cat{A}$ is a parallel pair
            \begin{tikzcd}
                X \arrow[r, shift left, "s"] \arrow[r, shift right, "t"'] & Y
            \end{tikzcd} of maps in $\cat{A}$.\par
            A cone on $D$ consists of objects and maps
            \begin{equation*}
            \begin{tikzcd}
                & A \arrow[dl, "f"'] \arrow[dr, "g"] \\
                X \arrow[rr, shift left, "s"] \arrow[rr, shift right, "t"'] && Y
            \end{tikzcd}
            \end{equation*}
            such that $s\circ f=g$ and $t\circ f=g$. But since $g$ is determined by $f$, it is equivalent to say that a cone on $D$ consists of an object $A$ and a map $f:A\to X$ such that
            \begin{equation*}
            \begin{tikzcd}
                A \arrow[r, "f"] & X \arrow[r, shift left, "s"] \arrow[r, shift right, "t"'] & Y
            \end{tikzcd}
            \end{equation*}
            is a fork. A limit of $D$ is an equalizer of $s$ and $t$.
        \item A diagram $D$ of shape $\bcat{P}$ in $\cat{A}$ consists of objects and maps
            \begin{equation*}
            \begin{tikzcd}
                & Y \arrow[d, "t"]\\
                X \arrow[r, "s"'] & Z
            \end{tikzcd}
            \end{equation*}
            in $\cat{A}$. A cone on $D$ is a commutative square. A limit of $D$ is a pullback.
    \end{enumerate}

    In general, the limit of a diagram $D$ is the terminal object in the category of cones on $D$.
\end{example}

\begin{definition}
    \begin{enumerate}[label=(\alph*)]
        \item Let $\bcat{I}$ be a small category. A category $\cat{A}$ \textbf{has limits of shape} $\bcat{I}$ if for every diagram $D$ of shape $\bcat{I}$ in $\cat{A}$, a limit of $D$ exists.
        \item A category \textbf{has all limits} if it has limits of shape $\bcat{I}$ for all small categories $\bcat{I}$.
    \end{enumerate}
\end{definition}

$\bcat{Set}, \bcat{Top}, \bcat{Grp}, \bcat{Ring}, \bcat{Vect}_k,\dots$ all have all limits. By definition, a category is \textbf{finite} if it contains only finitely many maps (in which case it also contains only finitely many objects). A \textbf{finite limit} is a limit of shape $\bcat{I}$ for some finite category $\bcat{I}$. For instance, binary products are finite limits.

\begin{proposition}\label{prop:limitsexistence}
    Let $\cat{A}$ be a category.
    \begin{enumerate}[label=(\alph*)]
        \item\label{prop:alllimits} If $\cat{A}$ has all products and equalizers then $\cat{A}$ has all limits.
        \item\label{prop:allfinitelimits} If $\cat{A}$ has binary products, a terminal object and equalizers then $\cat{A}$ has finite limits.
    \end{enumerate}
\end{proposition}

\begin{example}
    Recall that kernels provide equalizers in $\bcat{Vect}_k$. Then by \ref{prop:limitsexistence}\ref{prop:allfinitelimits} finite limits in $\bcat{Vect}_k$ can always be expressed in terms of $\oplus$ (binary direct sum), $\{0\}$, and kernels.
\end{example}

\begin{center}
    \textbf{Monics}
\end{center}
For maps in an arbitrary category, injectivity does not make sense, but there is a concept which is closely related.

\begin{definition}
    Let $\cat{A}$ be a category. A map $X\xrightarrow{f}Y$ in $\cat{A}$ is \textbf{monic} (or a \textbf{monomorphism}) if for all objects $A$ and maps \begin{tikzcd}
        A \arrow[r, shift left, "x"] \arrow[r, shift right, "x'"'] & X
    \end{tikzcd},
    \begin{equation*}
        f\circ x=f\circ x' \Rightarrow x=x'.
    \end{equation*}
\end{definition}

\begin{example}
    In $\bcat{Set}$, a map is monic if and only if it is injective. Indeed, if $f$ is injective then $f$ is monic, and for the converse, take $A=1$.
\end{example}

\begin{lemma}
    A map $X\xrightarrow{f}Y$ is monic if and only if the square
    \begin{equation*}
    \begin{tikzcd}
        X \arrow[r, "1"] \arrow[d, "1"'] & X \arrow[d, "f"] \\
        X \arrow[r, "f"'] & Y
    \end{tikzcd}
    \end{equation*}
    is a pullback.
\end{lemma}
\begin{proof}
    Consider the diagram
    \begin{equation*}
    \begin{tikzcd}
        A \arrow[drr, bend left, "p_1"] \arrow[ddr, bend right, "p_2"'] \arrow[dr, dotted, "t"description] \\
        & X \arrow[r, "1"] \arrow[d, "1"'] & X \arrow[d, "f"] \\
        & X \arrow[r, "f"'] & Y
    \end{tikzcd}
    \end{equation*}
    Assume that $f\circ p_1=f\circ p_2$. This means that the outer square commutes. Because the inner square is a pullback, we must have a unique arrow from $A$ to $X$, call it $t$, such that $1\circ t=p_1$ and $1\circ t=p_2$. This means $p_1=p_2$, hence, $f$ is monic.
\end{proof}

\subsection{Colimits}
\begin{definition}
    Let $\cat{A}$ be a category and $\bcat{I}$ a small category. Let $D:\bcat{I}\to\cat{A}$ be a diagram in $\cat{A}$, and write $D^{\text{op}}$ for the corresponding functor $\bcatop{I}\to\catop{A}$. A \textbf{cocone} on $D$ is a cone on $D^{\text{op}}$, and a \textbf{colimit} of $D$ is a limit of $D^{\text{op}}$.
\end{definition}

A cocone on $D$ is an object $A\in\cat{A}$ together with a family
\begin{equation*}
    \Big( D(I)\xrightarrow{f_I} A \Big)_{I\in\bcat{I}}
\end{equation*}
of maps in $\cat{A}$ such that for all maps $I\xrightarrow{u}J$ in $\bcat{I}$, the diagram
\begin{equation*}
\begin{tikzcd}
    D(I) \arrow[dr, "f_I"] \arrow[dd, "Du"'] \\
    & A \\
    D(J) \arrow[ur, "f_J"']
\end{tikzcd}
\end{equation*}
commutes. A colimit of $D$ is a cocone
\begin{equation*}
    \Big( D(I)\xrightarrow{p_I} C \Big)_{I\in\bcat{I}}
\end{equation*}
with the property that for any cocone on $D$, there is a unique map $\bar{f}:C\to A$ such that $\bar{f}\circ p_I=f_I$ for all $I\in\bcat{I}$. The maps $p_I$ are called coprojections.

\begin{center}
    \textbf{Sums}
\end{center}
\begin{definition}
    A \textbf{sum} or \textbf{coproduct} is a colimit over a discrete category.
\end{definition}
Let $(X_i)_{i\in I}$ be a family of objects of a category. Their sum is written as $\coprod_{i\in I}X_i$.\par

The sum of an empty set is an initial object.

\begin{example}
    Let $X_1$ and $X_2$ be vector spaces. There are linear maps
    \begin{equation}\label{eq:colimitvect}
        X_1 \xrightarrow{i_1} X_1 \oplus X_2 \xleftarrow{i_2} X_2
    \end{equation}
    defined by $i_1(x_1)=(x_1,0)$ and $i_2(x2)=(0,x_2)$. It can be checked that (\ref{eq:colimitvect}) are sums in the categorical sense. We have also seen that $X_1\oplus X_2$ is the product of $X_1$ and $X_2$ as well.
\end{example}

\begin{example}
    \textbf{Upper bounds} and \textbf{least upper bounds} in an ordered set, are sums in the corresponding category. In $(\mathbb{R},\leq)$, join is supremum and there is no least element. In $(\mathbb{N},\vert)$, join is lowest common multiple and the least element is $1$.
\end{example}

\begin{center}
    \textbf{Coequalizers}
\end{center}
We continue to write $\bcat{E}$ for the category $\bullet \rightrightarrows \bullet$.
\begin{definition}
    A \textbf{coequalizer} is a colimit of shape $E$.
\end{definition}
In other words, given a diagram \begin{tikzcd}
    X \arrow[r, shift left, "s"] \arrow[r, shift right, "t"'] & Y
\end{tikzcd}, a coequalizer of $s$ and $t$ is a map $Y\xrightarrow{p}C$ satisfying $p\circ s=p\circ t$ and universal with this property.\par

Coequalizers are something like quotients.

\begin{example}
    Take sets and functions \begin{tikzcd}
        X \arrow[r, shift left, "s"] \arrow[r, shift right, "t"'] & Y
    \end{tikzcd}. To find the coequalizer of $s$ and $t$, we must construct in some canonical way a set $C$ and a function $p:Y\to C$ such that $p(s(x))=p(t(x))$ for all $x\in X$. Let $\sim$ be the equivalence relation on $Y$ generated by $s(x)\sim t(x)$ for all $x\in X$. Take the quotient map $p:Y\to Y/\sim$. The maps $Y/\sim \to B$ correspond one-to-one with maps $f:Y\to B$, so $p$ is indeed the coequalizer of $s$ and $t$.
\end{example}

\begin{center}
    \textbf{Pushouts}
\end{center}
\begin{definition}
    A \textbf{pushout} is a colimit of shape
    \begin{equation*}
    \begin{tikzcd}
        X \arrow[r, "s"] \arrow[d, "t"] & Y \\
        Z
    \end{tikzcd}
    \end{equation*}
\end{definition}

\begin{example}
    Take such a diagram in $\bcat{Set}$. Its pushout $P$ is $(Y+Z)/\sim$, where $\sim$ is the equivalence relation on $Y+Z$ generated by $s(x)\sim t(x)$ for all $x\in X$. The coprojection $Y\to P$ sends $y\in Y$ to its equivalence class in $P$, and similarly for the coprojection $Z\to P$.
\end{example}

\begin{example}
    If $X=0$, where $0$ is an initial object, then the pushout is exactly a sum of $Y$ and $Z$.
\end{example}

\begin{center}
    \textbf{Epics}
\end{center}
\begin{definition}
    Let $\cat{A}$ be a category. A map $X\xrightarrow{f}Y$ in $\cat{A}$ is \textbf{epic} if for all objects $Z$ and maps \begin{tikzcd}
        Y \arrow[r, shift left, "g"] \arrow[r, shift right, "g'"'] & Z
    \end{tikzcd},
    \begin{equation*}
        g\circ f=g'\circ f \Rightarrow g=g'.
    \end{equation*}
\end{definition}

\begin{example}
    In $\bcat{Set}$, a map is epic if and only if it is surjective.
\end{example}
Any isomorphism in any category is both monic and epic. The converse is not always true, but it does hold in $\bcat{Set}$.

\subsection{Interactions between functors and limits}
\begin{definition}
    \begin{enumerate}[label=(\alph*)]
        \item\label{item:a} Let $\bcat{I}$ be a small category. A functor $F:\cat{A} \to \cat{B}$ \textbf{preserves limits of shape} $\bcat{I}$ if for all diagrams $D:\bcat{I}\to\cat{A}$ and all cones $\Big( A\xrightarrow{p_I} D(I) \Big)_{I\in\bcat{I}}$ on $D$,
            \begin{align*}
                \Big( A\xrightarrow{p_I} D(I) \Big)_{I\in\bcat{I}} \text{ is a limit cone on } D \text{ in } \cat{A}\\
                \Rightarrow \Big( F(A) \xrightarrow{Fp_I} FD(I) \Big)_{I\in\bcat{I}} \text{ is a limit cone on } F\circ D \text{ in } \cat{B}.
            \end{align*}
        \item A functor $F:\cat{A}\to\cat{B}$ \textbf{preserves limits} if it preserves limits of shape $\bcat{I}$ for all small categories $\bcat{I}$.
        \item \textbf{Reflection} of limits is defined as in \ref{item:a}, but with $\Leftarrow$ instead of $\Rightarrow$.
    \end{enumerate}
\end{definition}

\begin{example}
    The forgetful functor $U:\bcat{Top}\to\bcat{Set}$ preserves both limits and colimits, but it does not reflect all of them. For instance, choose two topological spaces $X$ and $Y$ which are not-discrete. Let $Z$ be the set $U(X)\times U(Y)$ equipped with the discrete topology. Then we have a cone $X\leftarrow Z\rightarrow Y$ in $\bcat{Top}$ whose image in $\bcat{Set}$ is a product cone, i.e., limit
    \begin{equation*}
        U(X)\leftarrow U(X)\times U(Y)\rightarrow U(Y).
    \end{equation*}
    But $Z$ is not a product cone, because the discrete topology on $U(X)\times U(Y)$ is not the product topology (this explains why we need the spaces $X$ and $Y$ to be non-discrete).
\end{example}

\begin{definition}
    A functor $F:\cat{A}\to\cat{B}$ \textbf{creates limits (of shape $\bcat{I}$)} if whenever $D:\bcat{I}\to\cat{A}$ is a diagram in $\cat{A}$,
    \begin{itemize}
        \item for any limit cone $\Big( B\xrightarrow{q_I} FD(I) \Big)_{I\in\bcat{I}}$ on the diagram $F\circ D$, there is a unique cone $\Big( A\xrightarrow{p_I} D(I) \Big)_{I\in\bcat{I}}$ on $D$ such that $F(A)=B$ and $F(p_I)=q_I$ for all $I\in\bcat{I}$;
        \item this cone $\Big( A\xrightarrow{p_I} D(I) \Big)_{I\in\bcat{I}}$ is a limit cone on $D$.
    \end{itemize}
\end{definition}

The forgetful functors from $\bcat{Grp}, \bcat{Ring},\dots$ to $\bcat{Set}$ all create limits.

\begin{lemma}
    Let $F:\cat{A}\to\cat{B}$ be a functor and $\bcat{I}$ a small category. Suppose that $\cat{B}$ has, and $F$ creates, limits of shape $\bcat{I}$. Then $\cat{A}$ has, and $F$ preserves limits of shape $\bcat{I}$.
\end{lemma}
