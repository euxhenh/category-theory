\section{Adjoints}
\subsection{Definition and examples}
Consider a pair of functors in opposite directions, $F:\cat{A}\to\cat{B}$ and $G:\cat{B}\to\cat{A}$. Roughly speaking, $F$ is said to be a left adjoint to $G$ if, whenever $A\in\cat{A}$ and $B\in\cat{B}$, maps $F(A)\to B$ are essentially the same as maps $A\to G(B)$.

\begin{definition}
    Let \begin{tikzcd}
        \cat{A} \arrow[r, shift left, "F"] & \cat{B} \arrow[l, shift left, "G"] 
    \end{tikzcd} be categories and functors. We say that $F$ is \textbf{left adjoint} to $G$, and $G$ is \textbf{right adjoint} to $F$, and write $F\dashv G$, if
    \begin{equation}\label{eq:adjoint_def}
        \cat{B}(F(A), B)\cong \cat{A}(A, G(B))
    \end{equation}
    naturally in $A\in\cat{A}$ and $B\in\cat{B}$.
\end{definition}
Naturally in $A\in\cat{A}$ and $B\in\cat{B}$ means that there is a specified bijection for each $A\in\cat{A}$ and $B\in\cat{B}$, and that it satisfies a naturality axiom. To state it, we need some notation. Given objects $A\in\cat{A}$ and $B\in\cat{B}$, the correspondence between maps $F(A)\to B$ and $A\to G(B)$ is denoted by a horizontal bar, in both directions:
\begin{align*}
    \big(F(A)\xrightarrow{g}B \big) &\mapsto \big(A\xrightarrow{\bar{g}}G(B) \big)\label{eq:nat1},\\
    \big(F(A)\xrightarrow{\bar{f}}B \big) &\mapsfrom \big(A \xrightarrow{f} G(B) \big).
\end{align*}
So $\bar{\bar{f}}=f$ and $\bar{\bar{g}}=g$. We call $\bar{f}$ the \textbf{transpose} of $f$, and similarly for $g$. The naturality axiom has two parts:
\begin{equation}\label{eq:naturality1}
    \overline{\big( F(A)\xrightarrow{g} B\xrightarrow{q} B' \big)} = \big( A\xrightarrow{\bar{g}} G(B) \xrightarrow{G(q)} G(B') \big)
\end{equation}
(that is, $\overline{q\circ g}=G(q)\circ\bar{g}$) for all $g$ and $q$, and
\begin{equation}\label{eq:naturality2}
    \overline{\big(A' \xrightarrow{p} A \xrightarrow{f} G(B) \big)} = \big(F(A') \xrightarrow{F(p)} F(A) \xrightarrow{\bar{f}} B\big)
\end{equation}
for all $p$ and $f$.\par
    A given functor $G$ may or may not have a left adjoint, but if it does, it is unique up to isomorphism. Same goes for right adjoints.

\begin{example}
    Forgetful functors between categories of algebraic structures usually have left adjoints. For instance:
    \item Let $k$ be a field. There is an adjunction
        \begin{equation*}
        \begin{tikzcd}
            \bcat{Vect}_k \arrow[d, shift left=2, "U"] \\
            \bcat{Set,} \arrow[u, shift left=3, "F", "\dashv"'] \
        \end{tikzcd}
        \end{equation*}
    where $U$ is the forgetful functor and $F$ is the free functor. Adjointness says that given a set $S$ and a vector space $V$, a linear map $F(S)\to V$ is essentially the same thing as a function $S\to U(V)$.\par
    Fix a set $S$ and a vector space $V$. Given a linear map $g:F(S)\to V$, we may define a map of sets $\bar{g}: S\to U(V)$ by $\bar{g}(s)=g(s)$ for all $s\in S$. This gives a function
    \begin{align*}
    \bcat{Vect}_k(F(S),V) &\to \bcat{Set}(S,U(V))\\
        g &\mapsto \bar{g}.
    \end{align*}
    In the other direction, given a map of sets $f:S\to U(V)$, we may define a linear map $\bar{f}:F(S)\to V$ by $\bar{f}\Big( \sum_{s\in S}\lambda_s s\Big) =\sum_{s\in S}\lambda_sf(s)$ for all formal linear combinations $\sum \lambda_s s\in F(S)$. This gives a function
    \begin{align*}
        \bcat{Set}(S, U(V)) &\to \bcat{Vect}_k(F(S), V) \\
        f &\mapsto \bar{f}.
    \end{align*}
    These two functions 'bar' are mutually inverse: for any linear map $g:F(S)\to V$, we have
    $$\bar{\bar{g}}\Bigg( \sum\limits_{s\in S} \lambda_s s \Bigg) = \sum\limits_{s\in S} \lambda_s \bar{g}(s) = \sum\limits_{s\in S} \lambda_s g(s) = g\Bigg( \sum\limits_{s\in S} \lambda_s s \Bigg)$$
    for all $\sum \lambda_s s \in F(S)$, so $\bar{\bar{g}}=g$, and for any map of sets $f: S\to U(V)$, we have
    $$\bar{\bar{f}}(s)=\bar{f}(s)=f(s)$$
    for all $s\in S$, so $\bar{\bar{f}}=f$. We therefore have a canonical bijection between $\bcat{Vect}_k(F(S), V)$ and $\bcat{Set}(S, U(V)$ for each $S\in \bcat{Set}$ and $V\in \bcat{Vect}_k$, as required.
\end{example}
\begin{remark}
    Note that we are using $(\bar{\cdot})$ for both directions. With a little care, the notation becomes clear and useful.
\end{remark}
\begin{example}
    There are adjunctions
    \begin{equation*}
    \begin{tikzcd}
        \bcat{Top} \arrow[dd, "U" description] \\ \\
        \bcat{Set} \arrow[uu, shift left=6, "D", "\dashv"'] \arrow[uu, shift right=6, "I"', "\dashv"]
    \end{tikzcd}
    \end{equation*}
    where $U$ sends a space to its set of points, $D$ equips a set with the discrete topology, and $I$ equips a set with the indiscrete topology.
\end{example}
\begin{example}
    Given sets $A$ and $B$, we can form their (cartesian) product $A\times B$. We can also form the set $B^A$ of functions from $A$ to $B$. This is the same as the set $\textbf{Set}(A,B)$, but we tend to use the notattion $B^A$ when we want to emphasize that it is an object of the same category as $A$ and $B$.\par
    Now fix a set $B$. Taking the product with $B$ defines a functor
    \begin{align*}
        \_\times B: \bcat{Set} &\to \bcat{Set} \\
        A &\mapsto A\times B.
    \end{align*}
    There is also a functor
    \begin{align*}
        (\_)^B: \bcat{Set} &\to \bcat{Set} \\
        C &\mapsto C^B.
    \end{align*}
    Moreover, there is a canonical bijection
    $$ \textbf{Set}(A\times B, C) \cong \textbf{Set}(A, C^B)$$
    for any sets $A$ and $C$. It is defined by simply changing the punctuation: given a map $g:A\times B\to C$, define $\bar{g}: A\to C^B$ by
    $$(\bar{g}(a))(b) = g(a,b)$$
    $(a\in A, b\in B)$, and in the other direction, given $f:A\to C^B$, define $\bar{f}:A\times B\to C$ by
    $$\bar{f}(a,b)=(f(a))(b)$$
    $(a\in A, b\in B)$. Putting all this together, we obtain an adjunction
    \begin{equation*}
    \begin{tikzcd}
        \bcat{Set} \arrow[d, shift left=2, "(\_)^B"] \\
        \bcat{Set} \arrow[u, shift left=2, "\_\times B", "\dashv"']
    \end{tikzcd}
    \end{equation*}
    for every set $B$.
\end{example}

\begin{definition}
    Let $\cat{A}$ be a category. An object $I\in\cat{A}$ is \textbf{initial} if for every $A\in\cat{A}$, there is exactly one map $I\to A$. An object $T\in\cat{A}$ is \textbf{terminal} if for every $A\in\cat{A}$, there is exactly one map $A\to T$.
\end{definition}
    For example, the empty set is initial in $\bcat{Set}$, the trivial group is initial in $\bcat{Grp}$, and $\mathbb{Z}$ is initial in $\bcat{Ring}$. The one-element set is terminal in $\bcat{Set}$, the trivial group is terminal (as well as initial) in $\bcat{Grp}$, and the trivial (one-element) ring is terminal in $\bcat{Ring}$.\par
    A category need not have an initial object, but if it does have one, it is unique up to isomorphism.
\begin{lemma}
    Let $I$ and $I'$ be initial objects of a category. Then there is a unique isomorphism $I\to I'$. In particular, $I\cong I'$.
\end{lemma}
\begin{proof}
    Since $I$ is initial, there is a unique map $f: I\to I'$. Since $I'$ is initial, there is a unique map $f':I'\to I$. Now $f'\circ f$ and $1_I$ are both maps $I\to I$, and $I$ is initial, so $f'\circ f=1_I$. Similarly, $f\circ f'=1_{I'}$. Hence, $f$ is an isomorphism.
\end{proof}

\begin{example}
    Initial and terminal objects can be described as adjoints. Let $\cat{A}$ be a category. There is precisely one functor $\cat{A} \to \textbf{1}$. Also, a functor $ \textbf{1}\to\cat{A}$ is essentially just an object of $\cat{A}$. Viewing functors $ \textbf{1}\to\cat{A}$ as objects of $\cat{A}$, a left adjoint to $\cat{A}\to \textbf{1}$ is exactly an initial object of $\cat{A}$.\par
    Similarly, a right adjoint to the unique functor $\cat{A}\to \textbf{1}$ is exactly a terminal object of $\cat{A}$.
\end{example}

\begin{remark}
    Adjunctions can be composed. Take adjunctions
    \begin{equation*}
    \begin{tikzcd}
        \cat{A}
        \arrow[r, shift left=2, "F"] &
        \cat{A'}
        \arrow[l, shift left=2, "G", "\bot"'] \arrow[r, shift left=2, "F'"] &
        \cat{A''.}
        \arrow[l, shift left=2, "G'", "\bot"']
    \end{tikzcd}
    \end{equation*}
    Then we obtain an adjunction
    \begin{equation*}
    \begin{tikzcd}
        \cat{A} \arrow[r, shift left=2, "F'\circ F"] &
        \cat{A''.} \arrow[l, shift left=2, "G\circ G'", "\bot"'] \
    \end{tikzcd}
    \end{equation*}
\end{remark}

\subsection{Adjunctions via units and counits}
To start building the theory of adjoint functors, we have to take seriously the naturality reqiurement. Suppose we have maps
\begin{equation*}
\begin{tikzcd}
    F(A) \arrow[r, "g"] & B \arrow[r, "q"] & B'
\end{tikzcd}
\end{equation*}
in $\cat{B}$. We can either compose and take the transpose, which produces a map $\overline{q\circ g}: A\to G(B')$, or take the trasnpose of $g$ then compose it with $G(q)$, which produces $G(q)\circ\bar{g}: A\to G(B')$. The first naturality equation says that they are equal.\par
For each $A\in\cat{A}$, we have a map
\begin{equation}\label{eq:eta}
    \big(A\xrightarrow{\eta_A} GF(A)\big) = \overline{\big(F(A)\xrightarrow{1} F(A)\big)}.
\end{equation}
Dually, for each $B\in\cat{B}$, we have a map
\begin{equation}\label{eq:eps}
    \big(FG(B)\xrightarrow{\epsilon_B} B\big) = \overline{\big(G(B)\xrightarrow{1} G(B)\big)}.
\end{equation}
These define natural transformations
\begin{equation*}
    \eta: 1_{\cat{A}}\to G\circ F,\qquad\epsilon: F\circ G\to 1_{\cat{B}},
\end{equation*}
called the \textbf{unit} and \textbf{counit} of the adjunction, respectively.

\begin{example}
    Take the usual adjunction \begin{tikzcd}
        \bcat{Vect}_k \arrow[r, shift left=2, "U"] &
        \bcat{Set} \arrow[l, shift left=2, "F", "\top"']
    \end{tikzcd}. Its unit $\eta: 1_{\bcat{Set}}\to U\circ F$ has components
    \begin{align*}
        \eta_S : S &\to UF(S) = \{\text{formal $k$-linear sums} \sum_{s\in S}\lambda_s s\} \\
        s & \mapsto \qquad s
    \end{align*}
    ($S\in\bcat{Set}$). The component of the counit $\epsilon$ at a vector space $V$ is the linear map
    $$\epsilon_V: FU(V)\to V$$
    that sends a \textit{formal} linear sum $\sum_{v\in V}\lambda_v v$ to its \textit{actual} value in $V$.\par
    The vector space $FU(V)$ is enormous. For instance, if $k=\mathbb{R}$ and $V$ is the vector space $\mathbb{R}^2$, then $U(V)$ is the set $\mathbb{R}^2$ and $FU(V)$ is a vector space with one basis element for every element of $\mathbb{R}^2$; thus, it is uncountably infinite-dimensional. Then $\epsilon_V$ is a map from this infinite-dimensional space to the 2-dimensional space $V$.
\end{example}

\begin{lemma}\label{lemma:adjunction}
    Given an adjunction $F\dashv G$ with unit $\eta$ and counit $\epsilon$, the triangles
    \begin{equation*}
    \begin{tikzcd}
        F \arrow[r, "F\eta"] \arrow[dr, "1_F"'] & FGF \arrow[d, "\epsilon F"] \\
        & F
    \end{tikzcd}
    \qquad
    \begin{tikzcd}
        G \arrow[r, "\eta G"] \arrow[dr, "1_G"'] & GFG \arrow[d, "G\epsilon"] \\
        & G
    \end{tikzcd}
    \end{equation*}
    commute.
\end{lemma}
\begin{remark}
    These are called the \textbf{triangle identities}. An equivalent statement is that the triangles
    \begin{equation}\label{lemma:triangle_identities}
    \begin{tikzcd}
        F(A) \arrow[r, "F(\eta_A)"] \arrow[dr, "1_{F(A)}"'] & FGF(A) \arrow[d, "\epsilon_{F(A)}"] \\
        & F(A)
    \end{tikzcd}
    \qquad
    \begin{tikzcd}
        G(B) \arrow[r, "\eta_{G(B)}"] \arrow[dr, "1_{G(B)}"'] & GFG(B) \arrow[d, "G(\epsilon_B)"] \\
        & G(B)
    \end{tikzcd}
    \end{equation}
    commute for all $A\in\cat{A}$ and $B\in\cat{B}$.
\end{remark}
\begin{proof}[Proof of Lemma \ref{lemma:adjunction}]
    We prove that the triangles (\ref{lemma:triangle_identities}) commute. Let $A\in\cat{A}$. From (\ref{eq:eps}) we know that $\overline{1_{GF(A)}}=\epsilon_{F(A)}$ and from (\ref{eq:eta}) we know that $\eta_A = \overline{1_{F(A)}}$, i.e., $\overline{\eta_A}=\overline{\overline{1_{F(A)}}}=1_{F(A)}$. We can express $\overline{\big( A\xrightarrow{\eta_A}GF(A)\big)}$ as $\overline{\big( A\xrightarrow{\eta_A} GF(A) \xrightarrow{1} GF(A) \big)}$ where we just extended $GF(A)$ by the identity arrow. Then from the second part of the naturality axiom (\ref{eq:naturality2}) we get
    \begin{equation*}
        \overline{\big( A\xrightarrow{\eta_A} GF(A) \xrightarrow{1} GF(A) \big)} = \big( F(A) \xrightarrow{F(\eta_A)} FGF(A) \xrightarrow{\epsilon_{F(A)}} F(A) \big).
    \end{equation*}
    But the left-hand side is $\overline{\eta_A}=1_{F(A)}$, proving the first identity. The second follows by duality.
\end{proof}

\begin{lemma}\label{lemma:adjunction_composition}
    Let \begin{tikzcd}
        \cat{A} \arrow[r, shift left=2, "F"] &
        \cat{B} \arrow[l, shift left=2, "G", "\bot"']
    \end{tikzcd} be an adjunction, with unit $\eta$ and counit $\epsilon$. Then
    \begin{equation*}
        \bar{g} = G(g)\circ \eta_A
    \end{equation*}
    for any $g: F(A)\to B$, and
    \begin{equation*}
        \bar{f} = \epsilon_B\circ F(f)
    \end{equation*}
    for any $f: A\to G(B)$.
\end{lemma}
\begin{proof}
    Let $g:F(A)\to B$. By using the naturality axiom (\ref{eq:naturality1}), we have
    \begin{align*}
        \overline{\big( F(A) \xrightarrow{g} B \big)} &= \overline{\big( F(A) \xrightarrow{1} F(A) \xrightarrow{g} B \big)} \\
                                                      &= \big( A \xrightarrow{\eta} GF(A) \xrightarrow{G(g)} G(B) \big)
    \end{align*}
    which proves the first statement. The second follows by duality.
\end{proof}

\begin{theorem}\label{thm:adj_nat}
    Take categories and functors \begin{tikzcd}
        \cat{A} \arrow[r, shift left=2, "F"] &
        \cat{B} \arrow[l, shift left=2, "G", "\bot"']
    \end{tikzcd}. There is a one-to-one correspondence between:
    \begin{enumerate}[label=(\alph*)]
        \item adjunctions between $F$ and $G$ (with $F$ on the left and $G$ on the right)
        \item pairs $\big( 1_{\cat{A}} \xrightarrow{\eta} GF, FG \xrightarrow{\epsilon} 1_{\cat{B}} \big)$ of natural transformations satisfying the triangle identities.
    \end{enumerate}
\end{theorem}

\begin{example}
    An adjunction between two ordered sets consists of order-preserving maps 
    \begin{tikzcd}
        A \arrow[r, shift left=1, "f"] &
        B \arrow[l, shift left=1, "g"]
    \end{tikzcd} such that
    \begin{equation}\label{eq:poset_adj}
        \forall a\in A, \forall b\in B, \qquad f(a)\leq b \Longleftrightarrow a\leq g(b).
    \end{equation}
    This is because each of the homsets in the definition of adjoints (\ref{eq:adjoint_def}) contains at most one element and for the isomorphism to hold, both homsets should either be empty or both contain a single arrow. Naturality axioms are trivially satisfied since any two maps with the same domain and codomain are equal in \textbf{PreOSet}.\par
    The unit of the adjunction says that $a\leq gf(a)$ for all $a\in A$, and the counit says that $fg(b)\leq b$ for all $b\in B$. Theorem (\ref{thm:adj_nat}) states that (\ref{eq:poset_adj}) is equivalent to:
    \begin{equation*}
        \forall a\in A, a\leq gf(a) \quad \text{and} \quad \forall b\in B, fg(b)\leq b.
    \end{equation*}
\end{example}

\begin{example}
    Let $X$ be a topological space. Take the set $\mathcal{C}(X)$ of closed subsets of $X$ and the set $\mathcal{P}(X)$ of all subsets of $X$, both ordered by $\subseteq$. There are order-preserving maps
    \begin{equation*}
        \begin{tikzcd}
            \mathcal{P}(X) \arrow[r, shift left=1, "\text{Cl}"] &
            \mathcal{C}(X) \arrow[l, shift left=1, "i"]
        \end{tikzcd}
    \end{equation*}
    where $i$ is the inclusion map and Cl is the closure. This is an adjunction, with Cl left adjoint of $i$ as can be seen by
    \begin{equation*}
        \text{Cl}(A) \subseteq B \Longleftrightarrow A\subseteq B
    \end{equation*}
    for all $A\subseteq X$ and closed $B\subseteq X$.
\end{example}

\begin{remark}
    An equivalence of categories is not necessarily an adjunction.
\end{remark}
\vspace{0.5cm}

The idea of an adjoint functor is best understood as an approximation of a possibly non-existent inverse. Any pair of adjoint functors, however, restricts to an equivalence of categories on subcategories. These subcategories are sometimes known as the \textbf{center of the adjunction}, and their objects are known as \textbf{fixed points} of the adjunction.\par
The equivalences of categories that arise from fixed points of adjunctions this way are often known as \textbf{dualities}.
\begin{definition}
    Let \begin{tikzcd}
        \cat{A} \arrow[r, shift left=2, "F"] &
        \cat{B} \arrow[l, shift left=2, "G", "\bot"']
    \end{tikzcd} be a pair of adjoint functors. Say that an object $A\in\cat{A}$ is a fixed point of the adjunction if its adjunction unit is an isomorphism
    \begin{equation*}
        A \xrightarrow[\cong]{\eta_A} GF(A)
    \end{equation*}
    and write $\cat{A}_{\text{fix}}$ for the full subcategory on these fixed objects. Similarly, an object $B\in\cat{B}$ is a fixed point of the adjunction if its adjunction unit is an isomorphism
    \begin{equation*}
        FG(B) \xrightarrow[\cong]{\epsilon_B} B
    \end{equation*}
    and write $\cat{B}_{\text{fix}}$ for the full subcategory on these fixed objects.
\end{definition}

\begin{proposition}
    The adjunction $(F,G,\eta,\epsilon)$ restricts to an adjoint equivalence $(F',G',\eta',\epsilon')$ on these full subcategories of fixed points
    \begin{equation*}
    \begin{tikzcd}
        \cat{A}_{\text{fix}} \arrow[r, shift left=2, "F'"] &
        \cat{B}_{\text{fix}} \arrow[l, shift left=2, "G'", "\bot"']
    \end{tikzcd}
    \end{equation*}
\end{proposition}
\begin{proof}
    The restricted adjunction unit/counit are isomorphisms by definition. So in order to show that the functors exhibit an adjoint equivalence, it is sufficient to see that the functors restrict as claimed.\par
    Hence we need to show that $F'(\cat{A}_{\text{fix}}) \subset  \cat{B}_{\text{fix}}$ and $G'(\cat{B}_{\text{fix}}) \subset \cat{A}_{\text{fix}}$. Consider the first of these.\par
    Given some $A\in\cat{A}_{\text{fix}}$ we need to show that $F'(A)\in\cat{B}_{\text{fix}}$, in other words we need to show that $\epsilon'_{F'(A)}$ is an isomorphism. But because $A\in\cat{A}_{\text{fix}}$, then $\eta'_A$ is an isomorphism. Functors preserve isomorphisms so $F'(\eta'_A)$ must be an isomorphism. Because inverses are unique and $\epsilon'_{F'(A)} \circ F'(\eta'_A) = 1_{F'(A)}$ by the triangle identities (\ref{lemma:triangle_identities}), the inverse of $F'(\eta'_A)$ must be $\epsilon'_{F'(A)}$ which means that $\epsilon'_{F'(A)}$ is itself an isomorphism.\par
    The second statement follows by duality.
\end{proof}

If the adjunction is idempotent, then the fixed objects in $\cat{A}$ are precisely those of the form $G(B)$, and dually the fixed objects in $\cat{B}$ are those of the form $F(A)$. This is essentially the definition of an idempotent adjunction.

\subsection{Adjunctions via initial objects}
This third formulation of adjointness is probably the most common.\par
Consider the adjunction
\begin{equation*}
\begin{tikzcd}
    \bcat{Vect}_k \arrow[d, shift left=2, "U", "\dashv"'] \\
    \bcat{Set.} \arrow[u, shift left=2, "F"]
\end{tikzcd}
\end{equation*}
Let $S$ be a set. The universal property of $F(S)$, the vector space whose basis is $S$, is most commonly stated like this:\par

\bigskip\noindent given a vector space $V$, any function $f:S\to V$ extends uniquely to a linear map $\bar{f}:F(S)\to V$.\par

\bigskip Forgetful functors are often forgotten: in this statement '$f:S\to V$' should strictly speaking be '$f:S\to U(V)$'. The word 'extends' refers to the embedding
\begin{align*}
    \eta_S:\quad & S\to UF(S)\\
            & s\mapsto s.
\end{align*}
In precise language the statement reads:\par\bigskip

    for any $V\in\bcat{Vect}_k$ and $f\in\bcat{Set}(S,U(V))$, there is a unique $\bar{f}\in\bcat{Vect}_k(F(S),V)$ such that the diagram
    \begin{equation}\label{eq:third_adj_def}
    \begin{tikzcd}
        S \arrow[r, "\eta_S"] \arrow[dr, "f"'] & U(F(S)) \arrow[d, "U(\bar{f})"] \\
                                              & U(V)
    \end{tikzcd}
    \end{equation}
    commutes.\par\bigskip

In this section we show that this statement is equivalent to the statement that $F$ is left adjoint to $U$ with unit $\eta$.

\begin{definition}
    Given categories and functors
    \begin{equation*}
    \begin{tikzcd}
        & \cat{B} \arrow[d, "Q"]\\
        \cat{A} \arrow[r, "P"'] & \cat{C},
    \end{tikzcd}
    \end{equation*}
    the \textbf{comma category} $\comma{P}{Q}$ (often written as $(P\downarrow Q)$) is the category defined as follows:
    \begin{itemize}
        \item objects are triples $(A,h,B)$ with $A\in\cat{A}, B\in\cat{B}$, and $h: P(A)\to Q(B)$ in $\cat{C}$;
        \item maps $(A,h,B)\to (A',h',B')$ are pairs $(f:A\to A', g:B\to B')$ of maps such that the square
            \begin{equation*}
            \begin{tikzcd}
                P(A) \arrow[r, "P(f)"] \arrow[d, "h"'] & P(A') \arrow[d, "h'"] \\
                Q(B) \arrow[r, "Q(g)"'] & Q(B')
            \end{tikzcd}
            \end{equation*}
            commutes.
    \end{itemize}
\end{definition}

\begin{example}
    Let $\cat{A}$ be a category and $A\in\cat{A}$. The \textbf{slice category} of $\cat{A}$ over $A$, denoted by $\cat{A}/A$, is the category whose objects are maps into $A$ and whose maps are commutative triangles. More precisely, an object is a pair $(X,h)$ with $X\in\cat{A}$ and $h:X\to A$ in $\cat{A}$, and a map $(X,h)\to (X',h')$ in $\cat{A}/A$ is a map $f: X\to X'$ in $\cat{A}$ making the triangle
    \begin{equation*}
    \begin{tikzcd}
        X \arrow[rr, "f"] \arrow[dr, "h"'] && X' \arrow[dl, "h'"] \\
        & A
    \end{tikzcd}
    \end{equation*}
    commute.
\end{example}
Slice categories are a special case of comma categories. Functors $\bcat{1}\to\cat{A}$ are just objects of $\cat{A}$. Now, given an object $A$ of $\cat{A}$, consider the comma category $\comma{1_{\cat{A}}}{A}$, as in the diagram
    \begin{equation*}
    \begin{tikzcd}
        & \bcat{1} \arrow[d, "A"] \\
        \cat{A} \arrow[r, "1_{\cat{A}}"'] & \cat{A}.
    \end{tikzcd}
    \end{equation*}
    An object of $\comma{1_{\cat{A}}}{A}$ is in principle a triple $(X,h,B)$, with $X\in\cat{A}$, $B\in\bcat{1}$, and $h:X\to A$ in $\cat{A}$; but $\bcat{1}$ has only one object, so it is essentially just a pair $(X,h)$. Hence the comma category $\comma{1_{\cat{A}}}{A}$ has the same objects as the slice category $\cat{A}/A$ and one can check that is has the same maps too, so $\cat{A}/A\cong \comma{1_{\cat{A}}}{A}$.\par

    Dually, there is \textbf{coslice category} $A/\cat{A}\cong \comma{A}{1_{\cat{A}}}$, whose objects are the maps out of $A$.\par

\begin{example}
    Let $G:\cat{B}\to\cat{A}$ be a functors and let $A\in\cat{A}$. We can form the comma category $\comma{A}{G}$, as in the diagram
    \begin{equation*}
    \begin{tikzcd}
        & \cat{B} \arrow[d, "G"] \\
        \bcat{1} \arrow[r, "A"'] & \cat{A}.
    \end{tikzcd}
    \end{equation*}
Its objects are pairs $(B\in\cat{B}, f:A\to G(B))$. A map $(B,f)\to (B',f')$ in $\comma{A}{G}$ is a map $q:B\to B'$ in $\cat{B}$ making the triangle
    \begin{equation*}
    \begin{tikzcd}
        A \arrow[r, "f"] \arrow[dr, "f'"'] & G(B) \arrow[d, "G(q)"] \\
                                           & G(B')
    \end{tikzcd}
    \end{equation*}
commute.

\end{example}
\bigskip We now make the connecton between comma categories and adjunctions.
\begin{lemma}\label{lemma:unit_is_initial}
    Take an adjunction \begin{tikzcd}
        \cat{A} \arrow[r, shift left=2, "F"] &
        \cat{B} \arrow[l, shift left=2, "G", "\bot"']
    \end{tikzcd} and an object $A\in\cat{A}$. Then the unit map $\eta_A: A\to GF(A)$ is an initial object of $\comma{A}{G}$.
\end{lemma}
\begin{proof}
    Take $(B,f)\in\comma{A}{G}$, where $B\in\cat{B}$ and $f:A\to G(B)$. We want to show that there exists a single map from $(F(A), \eta_A)$ to $(B,f)$. Such a map in $\comma{A}{G}$ is a map $q$ in $\cat{B}$ such that
    \begin{equation*}
    \begin{tikzcd}
        A \arrow[r, "\eta_A"] \arrow[dr, "f"'] & GF(A) \arrow[d, "G(q)"]\\
                                               & G(B)
    \end{tikzcd}
    \end{equation*}
    commutes. By Lemma \ref{lemma:adjunction_composition}, we know that $G(q)\circ\eta_A=\bar{g}$. So the triangle above commutes if and only if $\bar{g}=f$ if and only if $g=\bar{f}$. Hence $\bar{f}$ is the unique map from $(F(A),\eta_A)$ to $(B,f)$.
\end{proof}

This leads us the the third formulation of adjointness.

\begin{theorem}\label{thm:third_adj}
    Take categories and functors \begin{tikzcd}
        \cat{A} \arrow[r, shift left=2, "F"] &
        \cat{B} \arrow[l, shift left=2, "G", "\bot"']
    \end{tikzcd}. There is a one-to-one correspondence between:
    \begin{enumerate}[label=(\alph*)]
        \item adjunctions between $F$ and $G$;
        \item natural transformations $\eta:1_{\cat{A}}\to GF$ such that $\eta_A:A\to GF(A)$ is initial in $\comma{A}{G}$ for every $A\in\cat{A}$.
    \end{enumerate}
\end{theorem}
\begin{proof}
    Omitted.
\end{proof}

\begin{corollary}
    Let $G:\cat{B}\to\cat{A}$ be a functor. Then $G$ has a left adjoint if and only if for each $A\in\cat{A}$, the category $\comma{A}{G}$ has an initial object.
\end{corollary}
\begin{proof}
    We have already shown the 'only if' part from Lemma \ref{lemma:unit_is_initial}. Now choose for each $A\in\cat{A}$ an initial object of $\comma{A}{G}$ and call it $\big( F(A),\eta_A:A\to GF(A) \big)$. (Here $F(A)$ and $\eta_A$ are names we chose.) For each map $f:A\to A'$ in $\cat{A}$, let $F(f):F(A)\to F(A')$ be the unique map such that
    \begin{equation*}
    \begin{tikzcd}
        A \arrow[r, "\eta_A"] \arrow[d, "f"'] & G(F(A)) \arrow[d, "G(F(f))"]\\
        A' \arrow[r, "\eta_{A'}"'] & G(F(A'))
    \end{tikzcd}
    \end{equation*}
    commutes. It is easily checked that $F$ is a functor $\cat{A}\to\cat{B}$, and the diagram tells us that $\eta$ is a natural transformation $1\to GF$. By Theorem \ref{thm:third_adj}, $F$ is left adjoint to $G$.
\end{proof}
